\chapter{Acknowledgements}

\begin{ack}

Too many drafts of my acknowledgements I have mentally composed since
the conception of my thesis, and certainly none mentions a global
pandemic. Lately I think often of Erich Auerbach, who claims that, had
he not been exiled to wartime Istanbul with no research libraries for
him to consult, the overwhelming amount of source materials would have
prevented him from finishing \emph{Mimesis}. The thought---the phantom
rumblings of history beneath Auerbach's unruffled prose---gives me
comfort.

I thank, above all, my thesis advisor Professor Christian Thorne, whose
encyclopedic knowledge and unwavering enthusiasm carried me through
every moment of my Williams career. I have never looked at the world the
same way.

I thank Professor Anita Sokolsky and Professor Stephen Tifft, in matters
intellectual and otherwise. I could not have asked for a better pair of
academic parents: in the snack bar, Professor Sokolsky pushing over her
lunch plate, her eyes glittering in the wintry sun: ``Have a chip.''

I thank Professor Emily Vasiliauskas for being the final nail in my Princeton
coffin, Professor Christopher Pye for our feverish conversation on Lacan over
Tunnel City coffee, Professor John Limon for his encouragements during the
Honors Colloquium, Professor Anjuli Raza Kolb for inspiring the unwritten half
of this thesis. I thank Professor Nimu Njoya and Professor Jana Sawicki for all
our meditations between the individual and the world.

I thank all the fellows at the Oakley Center, especially Krista Birch,
Professor Gage McWeeny, and the Ruchman family. I thank Molly Magavern
and Robert Blay for caring so very much about their students.

The idea of adapting uneven development for a theory of culture, I
attribute, with gratitude, to Emma Lezberg; everything else, to
conversations with friends and mentors, in sleepless nights, in
shimmering light.

Kristen Altman, who is the most stringent Adornian I've encountered, who
once compared the taste of Tunnel's Red Line Blend to ``a hearty soup,''
whose infinite kindness shines through every blank space of my thesis,
and through the whole length of this couch on which she has tolerated my
endless scribbles---no amount of clauses, not even a run-on sentence,
can do her justice.

My thesis is dedicated to my parents, who sent their son this far, only
for to him return.

\vspace{5\baselineskip}
\raggedleft
Mi Yu \\
\emph{Killingworth, Connecticut}

\end{ack}

\begin{epi}
    \itshape
    \vspace*{5\baselineskip}
    But there was a time we were lashed to the prow \\
    Of a ship you may board, but not steer, \\
    Before you and I ceased to mean now, \\
    And began to mean only right here. \\

    \vspace{\baselineskip}
    \emph{Joanna Newsom, \emph{Waltz of the 101\textsuperscript{st} Lightborne}}
\end{epi}

\mainmatter
